\chapter{Угадыватель чисел}
\label{ch:guessnumbers}

\section{Аннотация} 

Данная глава посвящена основным этапам создания приложения «Угадыватель чисел», за основу которого взята задача, описанная в книге по математике, Л. Ф. Магницкого, «Арифметика» \cite{Galanin}.
В начале предлагается некоторое описание и решение задачи, затем рассказывается о процессе создания приложения. Подробно раскрываются такие темы, как работа с экранами приложения и создание процедур.
Также в тексте содержится некоторая информация о тестировании приложения.

\section{Описание задачи}

Суть предлагаемой нами задачи описана в старинной книге автора первого в России учебного пособия по математике, Л. Ф. Магницкого, «Арифметика», в главе: «Об утешных некиих действиях, через арифметику употребляемых» \cite{Galanin}.

Для начала предложим игроку загадать число, равное номеру любого дня недели. Дни недели пронумеруем от 1 (понедельник) до 7 (воскресенье). Далее попросим загадавшего выполнить следующие действия:

\begin{enumerate}
\item Умножить номер загаданного дня недели на 2.
\item К полученному произведению необходимо прибавить 5.
\item Затем полученную сумму умножить на 5.
\item Полученное число умножить на 10.
\item Назвать результат вычислений.
\end{enumerate}

Таким образом, мы легко сможем определить какое число загадал игрок.

\section{Решение задачи}

Раскроем секрет математического фокуса.
Чтобы перейти от полученного числа к загаданному числу, необходимо вычесть из него 250. Таким образом, мы получим число, в котором номер дня недели – число сотен. Рассмотрим доказательство решения задачи. Пусть {a} — искомое число (день недели). Выполним указанные действия над числом {a}:
\begin{enumerate}
   \item 2\cdota = 2a

   \item 2\cdota+5 = 2a + 5

   \item 5\cdot(2\cdota+5) = 10a + 25
 
   \item (10a+25)\cdot 10 = 100a + 250
 
   \item 100a+250 — 250 = 100a
\end{enumerate}

Таким образом, мы получим число, в котором номер дня недели — число сотен ({{a}})

\textbf{Задача:} В доказательстве алгоритма говорится, что  {a} — искомое число, но не уточняется, какое это число. Определите вид числа и его допустимые значения, приведите примеры.  Ответ на странице ... todo.

\section{Реализация приложения}

В данной программе \cite{PanfilovaApp} загаданный игроком номер дня недели это то, что мы будем искать.
Отметим, что обхекты имемнуются с помощью стиля Сamel Case \footnote[][-0cm]{\index{Процедуры и переменные!Именование} \emph{Именовать объекты} 
можно также произвольно, но рекомендуется использовать один из общепринятых стилей.
}\marginnote[0.2cm]{
Список самых распространенных стилей ~\ref{answer:naming} см. на с.~\pageref{answer:naming}.
}. 
Создадим глобальную переменную \footnote[][-0cm]{\index{Переменная!глобальная} \emph{Глобальная переменная} по-английски 
называется ``global variable''. К ней можно обратиться из любого места программы. 
}\marginnote[0.2cm]{
Чем хороши глобальные переменные и какие у них недостатки
по сравнению с локальными переменными? 
См. ответ~\ref{answer:global-vars-pros-cons} на с.~\pageref{answer:global-vars-pros-cons}.
}  \var{number} для хранения числа, загаданного игроком. Изначально присвоим переменной значение 0, а так как ноль не является допустимым загадываемым числом, переменная не будет хранить в себе число, которое мог бы загадать пользователь.

\subsection{Работа с экранами}
При проектировании приложения были учтены рекомендации из официальной документации App Inventor по ограничению количества экранов во избежание проблем с переполнением памяти \cite{MitManyScreens}.

Поэтому в игре используется восемь экранов (рекомендуемое количество < 10):
\begin{enumerate}
\item Screen1 — главный экран приложения представляющий из себя меню игры.
\item About — экран, содержащий основную информацию о приложении.
\item Steps — экран, который представляет собой описание последовательности действий, которые необходимо выполнить пользователю. На экране есть кнопки \footnote[][-0cm]{\index{Интерфейс пользователя!Button / Кнопка}
}\marginnote[0.2cm]{
Подробнее об элементе интерфейса «Кнопка» см. на с. TODO.
}. 
 \index{Интерфейс пользователя!Button / Кнопка} для переходов между описанием действий, изменяющих видимость элементов, тем самым визуально делается переход к определенному шагу.
\begin{enumerate}
  \item «На шаг назад» (previousStepButton) показывает описание предыдущего действия.
  \item «Далее» (nextStepButton) перемещает на шаг вперед.
  \item «В главное меню» (backToMainMenuButton) возвращает игрока на главный экран.
\end{enumerate}
\item FinalScreen — экран, на котором заканчивается игра. Здесь пользователю необходимо ввести получившееся в результате вычислений число и нажать кнопку «узнать ответ», чтобы приложение вывело на экран загаданное игроком число.

\end{enumerate}

\subsection{Тестирование}

\newthought{The front matter} of a book refers to all of the material that
comes before the main text.  The following table from shows a list of
material that appears in the front matter of \VDQI, \EI, \VE, and \BE
along with its page number.  Page numbers that appear in parentheses refer
to folios that do not have a printed page number (but they are still
counted in the page number sequence).
