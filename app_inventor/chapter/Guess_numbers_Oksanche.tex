\chapter{Угадыватель чисел}
\label{ch:guessnumbers}
\section{Описание задачи}
\newthought{ Суть предлагаемой нами задачи} описана в старинной книге автора первого в России учебного пособия по математике, Л. Ф. Магницкого, <<Арифметика>>, в главе: "Об утешных некиих действиях, через арифметику употребляемых".

Для начала предложим игроку загадать число, равное номеру любого дня недели. Дни недели пронумерованы от 1 (понедельник) до 7 (воскресенье). Далее попросим загадавшего выполнить следующие действия:

\begin{enumerate}
\item Умножить номер загаданного дня недели на 2.
\item К полученному произведению необходимо прибавить 5.
\item Затем полученную сумму умножить на 5.
\item Полученное число умножить на 10.
\item Назвать результат вычислений.
\end{enumerate}

Таким образом, мы легко сможем определить какое число загадал игрок.


\newthought{The front matter} of a book refers to all of the material that
comes before the main text.  The following table from shows a list of
material that appears in the front matter of \VDQI, \EI, \VE, and \BE
along with its page number.  Page numbers that appear in parentheses refer
to folios that do not have a printed page number (but they are still
counted in the page number sequence).
