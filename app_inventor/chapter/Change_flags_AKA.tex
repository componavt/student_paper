\chapter{Превращение флагов}
\label{ch:draughts-moves}

\newthought{В нашей задаче} будет дана стопка горизонтальных полосок разных цветов, 
из которых можно составить флаги государств. Сначала в игре показывают такие флаги и сообщают названия государств.
Игроку нужно расположить горизонтальные полоски так, чтобы они составили какой-либо известный флаг. 

\begin{marginfigure}
{%
\setlength{\fboxsep}{0pt}%
\setlength{\fboxrule}{1pt}%
\fcolorbox{gray}{gray}{\includegraphics{./commons/Flag_of_Republika_Srpska.png}}%
}%
%\includegraphics{./commons/Flag_of_Republika_Srpska.png}
%\fcolorbox{white}{gray}{\includegraphics{./commons/Flag_of_Republika_Srpska.png}}

\caption{Флаг Сербии. Какие ещё два флага можно получить, слегка меняя оттенок 
    и сдвигая полоски вверх или вниз? Ответы на с. ?? (Флаг России, флаг Крыма)}
\label{fig:Srpska}
\end{marginfigure}

\newthought{Сузим задачу}, ограничимся только трёхцветными флагами, как у России 
или Сербии (рис.~\ref{fig:Srpska}).
А по цветовой гамме выберем только флаги с белыми, синими или красными цветами.

\newthought{Сколько} таких флагов? И у каких государств? Ответы на эти вопросы 
можно найти с помощью Викиданных\marginnote{
Викиданные -- это компьютерная система, включающая базу данных, интерфейс для редактирования и язык запросов SPARQL для поиска в базе. Как и Википедию, Викиданные может редактировать каждый. Программисты любят Викиданные, потому что это та же Википедия, но в машиночитаемом виде, то есть язык Викиданных понимают и роботы, и компьютерные программы.
}. 
На Викиданных представлены \href{http://bit.ly/2OgIdWo}{328 государственных флага}\footnote{
    SPARQL-запрос к Викиданным для получения списка государственных флагов: \url{http://bit.ly/2OgIdWo}
}, 
причём с тремя горизонтальными полосками по версии Викиданных существует \href{http://bit.ly/2Q96ET1}{101 флаг}\footnote{
    Список горизонтальных триколоров по Викиданным
    \url{http://bit.ly/2Q96ET1}
}. 
Горизонтальных триколоров с красным, белым и синим цветом имеют всего \href{http://bit.ly/2xDTXsI}{19 государств}\footnote{
    Только синие, белые и/или красные цвета у триколоров 
    \url{http://bit.ly/2xDTXsI}
}. 

\newthought{Эти короткие ссылки}\marginnote{
    Короткие ссылки создаются специальными сайтами, которые берут длиннющую 
    гиперссылку и превращают её в коротенький набор букв и цифр. Это удобно, 
    если вы хотите оставить место в сообщении для своего текста и не тратить драгоценных знакомест на кашу из символов в URL.  
    Таких сервисов по сокращению URL много, bit.ly -- это один из них. 
    А почему сервис bitly называется bitly? Потому что ссылки становятся "a little BIT smaller", то есть "чуть-чуть покороче". 
    И потому что ``бит'' (англ. bit) -- это крошечное количество данных. 

} приведут вас на страницу ``Wikidata Query'', 
то есть на страницу запросов к базе Викиданных на языке SPARQL. 
Чтобы запустить запрос, нажмите на кнопку Play (рис.~\ref{fig:blue:button}). 
Для превращения скучного списка названий флагов в мозаику флагов, 
щёлкните по значку глаза, затем выберите пункт выпадающего меню ``Image grid'' (рис. ???).\sidenote[][-1.0cm]{This sidenote is 1 centimeter higher than it
normally would be and uses its original sidenote number.}

%
%play_blue_button_wikidata.png
%image_grid_select.png

\begin{figure}
  \includegraphics[width=0.4\columnwidth]{./wikidata/play_blue_button_wikidata.png}
%  \checkparity This is an \pageparity\ page.%
  \caption[Hilbert curves of various degrees $n$.][6pt]{Hilbert curves of various degrees $n$. \emph{Notice that this figure only takes up the main textblock width.}}
  \label{fig:blue:button}
  %\zsavepos{pos:textfig}
\end{figure}


Задачка: сколько ссылок можно закодировать, если ссылки содержит ровно 7 буквенно-цифровых символов,
     каждый символ может быть буквой английского алфавита 
     или цифрой от 0 до 9? Ответ на странице ... todo 
