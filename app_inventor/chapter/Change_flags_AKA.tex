\chapter{Превращение флагов}
\label{ch:draughts-moves}

\newthought{В нашей задаче} будет дана стопка горизонтальных полосок разных цветов, 
из которых можно составить флаги государств. Сначала в игре показывают такие флаги и сообщают названия государств.
Игроку нужно расположить горизонтальные полоски так, чтобы они составляли один из известных флагов. 

\begin{marginfigure}
{%
\setlength{\fboxsep}{0pt}%
\setlength{\fboxrule}{1pt}%
\fcolorbox{gray}{gray}{\includegraphics{./commons/Flag_of_Republika_Srpska.png}}%
}%
%\includegraphics{./commons/Flag_of_Republika_Srpska.png}
%\fcolorbox{white}{gray}{\includegraphics{./commons/Flag_of_Republika_Srpska.png}}

\caption{Флаг Сербии. Какие ещё два флага можно получить, слегка меняя оттенок 
    и сдвигая полоски вверх или вниз? Ответы на с. ?? (Флаг России, флаг Крыма)}
\label{fig:Srpska}
\end{marginfigure}

\newthought{Сузим задачу}, ограничимся только трёхцветными флагами, как у России 
или Сербии (рис.~\ref{fig:Srpska}).
По цветовой гамме выберем только те флаги, которые содержат белые, синии и красные цвета или какие-либо из этих цветов.

\newthought{Сколько} таких флагов? И у каких государств? Ответы на эти вопросы 
можно найти с помощью Викиданных\marginnote{
Викиданные -- это компьютерная система, включающая базу данных, интерфейс для редактирования и язык запросов SPARQL для поиска в базе. Как и Википедию, Викиданные может редактировать каждый. Программисты любят Викиданные, потому что это та же Википедия, но в машиночитаемом виде, то есть язык Викиданных понимают и роботы, и компьютерные программы.
}. 
На Викиданных представлены \href{http://bit.ly/2OgIdWo}{328 государственных флага}\footnote{
    SPARQL-запрос к Викиданным для получения списка государственных флагов: \url{http://bit.ly/2OgIdWo}.
}, 
причём с тремя горизонтальными полосками по версии Викиданных существует \href{http://bit.ly/2Q96ET1}{101 флаг}\footnote{
    \url{http://bit.ly/2Q96ET1}
}. 
Горизонтальных триколоров с красным, белым и синим цветом имеют всего \href{http://bit.ly/2xDTXsI}{19 государств}\footnote{
    \url{http://bit.ly/2xDTXsI}
}. 

\newthought{Эти короткие ссылки} приведут вас на страницу ``Wikidata Query'', то есть на страницу запросов к базе Викиданных на языке SPARQL. Чтобы запустить запрос, нажмите на кнопку Play.
