\chapter{Превращение флагов}
\label{ch:draughts-moves}

\newthought{В нашей задаче} будет дана стопка полосок разных цветов и 
можно предварительно или в ходе игры узнать, какие государства имеют трёхцветные флаги. 
Игроку нужно расположить последовательно полоски так, чтобы они составляли один из известных флагов. 
Выбраны флаги, которые содержат полоски белого, синего и красного цвета или все эти цвета сразу.

На Викиданных представлены \hyperref[328 государственных флага]{http://bit.ly/2OgIdWo}, 
%причём с тремя горизонтальными полосками по версии Викиданных существует \hyperref[101 флаг]{http://bit.ly/2Q96ET1}. 
%Горизонтальных триколоров с красным, белым и синим цветом имеют всего \hyperref[19 государств]{http://bit.ly/2xDTXsI}. 



\newthought{The front matter} of a book refers to all of the material that
comes before the main text.  The following table from shows a list of
material that appears in the front matter of \VDQI, \EI, \VE, and \BE
along with its page number.  Page numbers that appear in parentheses refer
to folios that do not have a printed page number (but they are still
counted in the page number sequence).
