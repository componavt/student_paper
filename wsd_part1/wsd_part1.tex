\documentclass{article}
\usepackage{krctran}

\begin{document}

\procname{Труды Карельского научного центра РАН\\ \No 0. 2010. С.~1--4}
\udk{УДК 004.01:006.72 (470.22)}

\rustitle{Обзор методов (уточнить - каких?) \newline по решению задачи разрешения лексической многозначности}
\engtitle{Word-sense disambiguation methods (specific?) review}

\rusauthor{С.~С.~Ткач}
\engauthor{S.~S.~Tkach}

\organization{Институт прикладных математических исследований Карельского научного центра РАН}

\rusabstract{Данный файл является примером статьи для научного издания Труды Карельского научного центра РАН, 
серия <<Математическое моделирование и информационные технологии>>.
В нем содержатся основные используемые переменные и окружения. При подготовке статьи рекомендуется
воспользоваться этим примером в качестве шаблона. Данный абзац оформлен в стиле аннотации статьи.}
\engabstract{This file is an auxiliary example of an article prepared forTransactions of Karelian Research Centre of RAS.
It contains most useful environments and variables. While preparing your article, it's recommended to use
this text as a template. This paragraph is formatted as an Abstract of the article.}

\ruskeywords{труды, шаблон, подготовка статьи.}
\engkeywords{transactions, template, article.}

\maketitle

\begin{articletext}
\section{Введение}
Статья, представляемая в научное издание, должна быть оформлена в соответствии
с <<Правилами для авторов>>, размещенными на сайте \hbox{http://transactions.krc.karelia.ru}.
Данный документ претендует на роль технической документации в помощь авторам.
Достаточно подробную информацию по наборе в системе \LaTeX можно найти,
напр., в работе~\cite{Latexrus}.

\section{Структура файла в формате \LaTeXe}

\begin{verbatim}
\documentclass{article}
\usepackage{krctran}
...
\begin{document}

\procname{...}
\udk{...}
\rustitle{...}
\engtitle{...}
\rusauthor{...}
\engauthor{...}
\organization{...}
\rusabstract{...}
\engabstract{...}
\ruskeywords{...}
\engkeywords{...}

\maketitle

\begin{articletext}
\section{...}
...
\begin{thebibliography}
...
\end{thebibliography}
\end{articletext}

\section{СВЕДЕНИЯ ОБ АВТОРE:}
\begin{aboutauthors}
...
\end{aboutauthors}
\end{document}
\end{verbatim}

Преамбула статьи должна содержать две обязательные команды:
\begin{verbatim}
\documentclass{article}
\usepackage{krctran}
\end{verbatim}

Далее формируется заголовок статьи: выходные данные, код УДК, название статьи, 
авторы с указанием мест работы, аннотация,
ключевые слова. Например, заголовок этого файла сформирован следующими командами: 
\begin{verbatim}
\procname{Труды...\\ \No...}
\udk{УДК...}
\rustitle{Руководство...}
\engtitle{Usage...}
\rusauthor{А.~С.~Румянцев}
\engauthor{A.~S.~Rumyantsev}
\organization{Институт...}
\rusabstract{Данный файл...}
\engabstract{This file...}
\ruskeywords{труды,...}
\engkeywords{transactions,...}
\maketitle
\end{verbatim}

\begin{Remark}
Для ручной разбивки на строки названия статьи воспользуйтесь командой \verb"\newline".
\end{Remark}

\bfullwidth
\begin{figure}
%\includegraphics[height=100mm,width=0.9\textwidth]{delay_HT.pdf}
\caption{Задержки в модели 10-узловой системы, случай тяжелых хвостов}
\label{fig5}
\end{figure}
\efullwidth

\section{Основной текст}
Текст статьи, а также список используемой литературы заключаются
в окружение \verb"{articletext}".

При наборе формул, желательно пользоваться
окружениями \verb"{equation}", \verb"{align}" и др.
Для ссылок на формулы удобно использовать команду \verb"\eqref{...}".

\section{Библиография}
Библиографические ссылки принято оформлять в виде [номер], в отличие от
ранее принятых [Автор, год] (см., напр., \cite{Trans}). 
Источник, процитированный выше, был набран командой
\begin{verbatim}
\bibitem{Trans}
\textit{Борисов~Г.~А., 
Тихомирова~Т.~А.}
Характеристики и свойства потерь 
энергии и мощности на пределах 
энергетического хозяйства региона //
Труды Карельского научного центра 
Российской академии наук. 2010. 
\No 3. С.~4--10.
\end{verbatim}

\section{Теоремоподобные окружения}
Для теорем, утверждений и пр. необходимо использовать соответствующие
окружения. Например:

\begin{State} 
В предложенной модели системы обслуживания при
$\rho=ES/ET<1$ условие $ES^{\alpha+1}<\infty$ является достаточным для 
конечности момента порядка $\alpha$ времени ожидания в системе, $ED^\alpha<\infty$.
\end{State}
\begin{proof}
Очевидно.
\end{proof}

В данном случае было использовано окружение \verb"\begin{State}...\end{State}". Для
набора доказательства использовалось окружение \verb"\begin{proof}...\end{proof}".
Доступные автору теоремоподобные окружения перечислены в Таблице~\ref{theorems}.

\begin{table}[H]
\centering
\caption{Теоремоподобные окружения}
\begin{tabular}{|l|l|}
\hline
\verb"Theorem" & Теорема\\
\verb"Lemma" & Лемма\\
\verb"State" & Утверждение\\
\verb"Corollary" & Следствие\\
\verb"Axiom" & Аксиома\\
\verb"Definition" & Определение\\
\verb"Example" & Пример\\
\verb"Remark" & Замечание\\
\hline
\end{tabular}
\label{theorems}
\end{table}

Для определений, примеров и замечаний используется прямое написание.
\begin{Example*}
Например, как в этом примере.
\end{Example*}

Соответствующие версии окружений <<со звездой>> также работают. Пример
выше был набран такой командой:

\begin{verbatim}
\begin{Example*}
Например, как в этом примере.
\end{Example*}
\end{verbatim}

\bfullwidth
\begin{table}[H]
\caption{Таблица, демонстрирующая возможность размещения на всю ширину страницы}
\begin{tabular}{|c|c|c|c|c|}
\hline
Первая колонка & вторая колонка & третья колонка & четвертая колонка & пятая колонна\\
\hline
\end{tabular}
\label{tab_width}
\end{table}
\efullwidth

\section{Рисунки и таблицы}
Рисунки и таблицы могут вставляться как на всю ширину страницы,
так и на ширину колонки. Желательно использовать рисунки формата pdf. 
Для конвертации из формата eps можно воспользоваться утилитой epstopdf.
Так, например, Рис.~\ref{fig1} был вставлен на ширину колонки командой
\begin{verbatim}
\begin{figure}[H]
\includegraphics[keepaspectratio=true,
 width=0.9\columnwidth]{delay_80.pdf}
\caption{Задержки в модели 
 10-узловой системы}
\label{fig1}
\end{figure}
\end{verbatim}

\begin{figure}[H]
    \includegraphics[keepaspectratio=true,width=0.9\columnwidth]{delay_80.pdf}
    \caption{Задержки в модели 10-узловой системы}
    \label{fig1}
\end{figure}



\section{Размещение на всю ширину страницы}
В стилевом файле предусмотрена возможность размещения формул, рисунков
и таблиц на ширину страницы. Для этого размещаемый элемент необходимо
заключить между командами \verb'\bfullwidth' и \verb'\efullwidth'.
 
Пример формулы на всю ширину страницы:
\bfullwidth
\begin{equation}
Y=A_1x+A_2x^2+\ldots +A_nx^n.
\end{equation}
\efullwidth
Набран пример следующим образом:
\begin{verbatim}
\bfullwidth
\begin{equation}
Y=A_1x+A_2x^2+\ldots +A_nx^n.
\end{equation}
\efullwidth
\end{verbatim}


Пример размещения таблицы на ширину страницы (см. таблицу~\ref{tab_width}):
\begin{verbatim}
\bfullwidth
\centering
\begin{table}[H]
\begin{tabular}{|c|c|c|c|c|}
\hline
Первая колонка & ...\\
\hline
\end{tabular}
\label{tab_width}
\end{table}
\efullwidth
\end{verbatim}

Рис.~\ref{fig5} демонстрирует возможности вставки по всей ширине страницы.
Это было достигнуто при помощи команды
\begin{verbatim}
\bfullwidth
\begin{figure}
\includegraphics[height=100mm,...}
\caption{Задержки в модели...}
\label{fig5}
\end{figure}
\efullwidth
\end{verbatim}
Следует обратить внимание на то, что вышеуказанная команда
вставит рисунок не ближе, чем на следующей странице сверху,
а не сразу на месте указания команды.

\section{Сведения об авторах}
После основного текста оформляются сведения об авторах. Используется
окружение \verb"\begin{aboutauthors}...\end{aboutauthors}". При этом следует обратить внимание, что
работа ведется в двухколоночном режиме, поэтому необходимо вручную 
указать разрыв колонки для отделения сведений на русском и английском
языках. Например:
\begin{verbatim}
\begin{aboutauthors}
\authorsname{Румянцев Александр...}
аспирант\\ 
...
\columnbreak
\authorsname{Rumyantsev, Alexander}
...
\end{aboutauthors}
\end{verbatim}

\section{Заключение}
Компиляцию исходного файла желательно выполнять с помощью макроса \verb"pdflatex".

В работе рассмотрены основные технические аспекты подготовки статьи
для сборника Трудов Карельского научного центра РАН. Предложения
и пожелания по доработке стилевого файла, а также текста этого документа
принимаются по электронному адресу, указанному в разделе <<Сведения об авторах>>.

\begin{thebibliography}{9}
%\bibitem[Борисов, Тихомирова(2010)]{Trans}
\bibitem{Trans}
\textit{Борисов~Г.~А., Тихомирова~Т.~А.} Характеристики и свойства потерь энергии
и мощности на пределах энергетического хозяйства региона //
Труды Карельского научного центра Российской академии наук. 2010. \No 3. С.~4--10.

\bibitem{Latexrus}
\textit{Львовский С. М.} Набор и верстка в системе \LaTeX. М., 2003. 448~с.
\end{thebibliography}
\end{articletext}


\section{СВЕДЕНИЯ ОБ АВТОРE:}
\begin{aboutauthors}
\authorsname{Румянцев Александр Cергеевич}
аспирант\\ 
Институт прикладных математических исследований КарНЦ РАН\\ 
ул. Пушкинская, 11, Петрозаводск, Республика Карелия, Россия, 185910\\
эл. почта: ar0@krc.karelia.ru\\
тел.: (8142) 763370

\columnbreak

\authorsname{Rumyantsev, Alexander}
Institute of Applied Mathematical Research, Karelian Research Centre, Russian Academy of Science\\
11 Pushkinskaya St., 185910 Petrozavodsk, Karelia, Russia\\
e-mail: ar0@krc.karelia.ru\\
tel.: (8142) 763370
\end{aboutauthors}
\end{document}
