% hello2.tex
\documentclass[12pt]{article} 
\usepackage[english,russian]{babel}
\usepackage[T2A]{fontenc} % Поддержка русских букв
\usepackage[utf8]{inputenc}


\title{Обзор статьи  <<Resolving Ambiguities in Biomedical Text With Unsupervised Clustering Approaches>>}
\author{ПетрГУ, 508 группа, Ярышкина Е.А.}
\date{11 ноября 2014 г.}


\begin{document} 
\maketitle

В статье \cite{Savova} изучаются уже существующие методы кластеризации без учителя и их эффективность для решения лексической многозначности при обработке текстов по биомедицине. Решение проблем лексической многозначности в данной области включает в себя не только традиционные задачи присвоения ранее определенных смысловых значений для терминов, но так же и обнаружения новых значений для них, ещё не включённых в данную онтологию.
\parindent=0,5cm

Авторы описали методологию метода решения лексической многозначности без учителя, учитываемые лексические признаки и наборы экспериментальных данных. В качестве оценки эффективности алгоритмов кластеризации текста была предложена F-мера. 
\parindent=0,5cm

Подход для решения поставленной задачи -- это разделение контекстов (фрагментов текста), содержащих определенное целевое слово на кластеры, где каждый кластер представляет собой различные значения целевого слова. Каждый кластер состоит из близких по значению контекстов. Задача решается в предположении, что используемое целевое слово в аналогичном контексте будет иметь один и тот же или очень похожий смысл. 
\parindent=0,5cm

Процесс кластеризации продолжается до тех пор, пока не будет найдено предварительно заданное число кластеров. В данной статье выбор шести кластеров основан на том факте, что это больше, чем максимальное число возможных значений любого английского слова, наблюдаемое среди данных (большинство слов имеют два-три значения). Нормализация текста не выполняется.
\parindent=0,5cm

Данные в этом исследовании состоят из ряда контекстов, которые включают данное целевое слово, где у каждого целевого слова вручную отмечено -- какое значение из словаря было использовано в этом контексте. Контекст -- это единственный источник информации о целевом слове. Цель исследования -- преобразовать контекст в контекстные вектора первого и второго порядка \cite{epr:website}. Контекстные вектора содержат следующие «лексические свойства»: биграммы, совместную встречаемость и совместную встречаемость целевого слова. Биграммами являются как двухсловные словосочетания, так и любые два слова, расположенные рядом в некотором тексте. Для лингвистических исследований могут быть полезны только упорядоченные наборы биграмм \cite{Averin}. 
\parindent=0,5cm

Экспериментальные данные -- это набор NLM WSD \cite{UMLS:website} (NLM -- национальная библиотека медицины США), в котором значения слов взяты из UMLS (единая система медицинской терминологии). UMLS имеет три базы знаний: 
\begin{itemize}
\item Метатезаурус включает все термины из контролируемых словарей (SNOMED-CT, ICD и другие) и понятия, которые представляют собой кластера из терминов, описывающих один и тот же смысл. 
\item Семантическая сеть распределяет понятия на 134 категории и показывает отношения между ними. SPECIALIST-лексикон содержит семантическую информацию для терминов Метатезауруса. 
\item Medline -- главная библиографическая база данных NLM, которая включает приблизительно 13 миллионов ссылок на журнальные статьи в области науки о жизни с уклоном в биомедицинскую область.
\end{itemize}
\parindent=0,5cm

Авторы успешно проверили по три конфигурации существующих методов (PB -- Pedersen and Bruce \cite{Pedersen}, SC -- Schütze \cite{Schutze}) и оценили эффективность использования SVD (сингулярное разложение матриц). Методы PB основаны на контекстных векторах первого порядка -- признаки одновременного присутствия целевого слова или биграммы. Рассчитывается среднее расстояние между кластерами или применяется  метод бисекций. PB методы подходят для работы с довольно большими наборами данных. Методы SC основаны на представлениях второго порядка -- матрицы признаков одновременного присутствия или биграммы, где каждая строка и столбец -- вектор признаков первого порядка данного слова. Так же рассчитывается среднее расстояние между кластерами или применяется  метод бисекций. SC методы подходят для обработки небольших наборов данных.  

\parindent=0,5cm
Метод SC2 (признаки одновременного присутствия второго порядка, среднее расстояние между элементами кластера в пространстве подобия) с применением и без SVD показал лучшие результаты: всего 56 сравниваемых экземпляров, в 47 случаях метод SC2 показал наилучшие результаты, в 7 случаях результаты незначительно отличаются от других проверяемых методов.
\parindent=0,5cm

Все эксперименты, указанные в исследовании, выполнялись с помощью пакета SenseClusters \cite{SC:website}. В ходе исследования было проведено два эксперимента для разных наборов данных. Маленький тренировочный набор -- это набор NLM WSD, который включает 5000 экземпляров для 50 часто встречаемых неоднозначных терминов из Метатезауруса UMLS. Каждый неоднозначный термин имеет по 100 экземпляров с указанным вручную значением. У 21 термина максимальное число экземпляров находится в пределах от 45 до 79 экземпляров. У 29 терминов число экземпляров от 80 до 100 для  конкретного значения. Стоит отметить, что каждый термин имеет категорию «ни одно из вышеупомянутых», которая охватывает все оставшиеся значения, не соответствующие доступным в UMLS. Большой тренировочный набор является реконструкцией «1999 Medline», который был разработан Weeber \cite{Weeber}. Были определены все формы из набора NLM WSD и сопоставлены с тезисами «1999 Medline». Для создания тренировочного набора экземпляров использовались только те тезисы из «1999 Medline», которым было найдено соответствие в наборе NLM WSD.
\parindent=0,5cm

Использование целиком текста аннотации статьи в качестве контекста приводит к лучшим результатам, чем использование отдельных предложений. С одной стороны, большой объем контекста, представленный аннотацией, дает богатую коллекцию признаков, с другой стороны, в коллекции WSD представлено небольшое число контекстов.

\addcontentsline{toc}{chapter}{\bibname}
\bibliographystyle{utf8gost705u}  %% стилевой файл для оформления по ГОСТу
\bibliography{bibl} 
Ярышкина Екатерина Александровна\\
Студентка\\
Математический факультет\\
Петрозаводский государственный университет\\
пр-кт Ленина, 33, Петрозаводск, Республика Карелия\\
+7 (8142) 71-10-78\\
kate.rysh@gmail.com\\
 \\
Yaryshkina Ekaterina Alexandrovna\\
Student\\
Faculty of Mathematics\\
Petrozavodsk State University\\
Prospect Lenina, 33, Petrozavodsk, Republic of Karelia\\
+7 (8142) 71-10-78\\
kate.rysh@gmail.com  
\end{document}